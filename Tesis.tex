% Created 2018-05-23 Wed 20:43
% Intended LaTeX compiler: pdflatex
\documentclass[twoside,openright]{book}
\usepackage[utf8]{inputenc}
\usepackage[T1]{fontenc}
\usepackage{graphicx}
\usepackage{grffile}
\usepackage{longtable}
\usepackage{wrapfig}
\usepackage{rotating}
\usepackage[normalem]{ulem}
\usepackage{amsmath}
\usepackage{textcomp}
\usepackage{amssymb}
\usepackage{capt-of}
\usepackage{hyperref}
\usepackage{lmodern}
\usepackage{mathtools}
\usepackage{url}
\usepackage{color}
\usepackage{amssymb}
\usepackage{amsopn}
\usepackage{nicefrac}
\usepackage{units}
\usepackage{gensymb}
\newcommand{\rojo}[1]{{\color{red}{#1}}}
\date{\today}
\title{}
\hypersetup{
 pdfauthor={Gerardo Guillermo},
 pdftitle={},
 pdfkeywords={},
 pdfsubject={},
 pdfcreator={Emacs 25.1.7 (Org mode 9.1.3)}, 
 pdflang={English}}
\begin{document}

\tableofcontents
\listoftables
\listoffigures

\chapter{Introduction}
\label{sec:org89a9126}
Synchrotron Radiation (SR) emission is the phenomenon product of the change of
direction of relativistic-charged particles.
Ever since it was observed in 1947 at the G.E. Synchrotron, the
interest in its study has increased, even though great part of the theory was
developed before that, and it is still being developed at the present (after a
hundred years). 

In electron storage rings and circular accelerators, the emitted synchrotron
radiation reaches high intensities and energies. Although,  Nowadays, there are dedicated
electron accelerators with the purpose of producing SR with specific
characteristics; in many other accelerators, S.R. is considered a secondary
product which is nocuous to the operation of the machine.

The power of emitted S.R. depends on the square of the Lorentz factor
(\(\gamma_r^{2}\)). For this reason S.R. was considered negligible in hadron
accelerators. It was until the construction of LHC, that hadron S.R. became a
problem for protons with a \(\gamma_r^{2}=7460\) that the negative effects of S.R. were not
negligible. The LHC is the coldest place in the universe, in order to keep the
superconductor magnetic coils at 1.9 °K, gigantic cryogenic plants are used
because of the arc S.R. heat load of 0.17 W/m/aperture. The vacuum in the LHC is
as low as interestellar vacuum \(~10^{-14}\) bar, high-energy photons from S.R. can
remove molecules and particles from the wall, degrading this vacuum.  Yet
another problem related to S.R. is the formation of electron clouds. These
clouds could grow from seed electrons torn from the wall by incident photons,
these electrons are attracted by the proton beam and also follow the magnetic
field lines. Electron-cloud problems are the main motivation for these studies.

The main objective of this work is to map the absorption points of S.R. photons in the
arcs of the LHC, with the idea that this map could be used in the future to
generate a photon distribution function (P.D.F.) to be used as an input for
electron-cloud simulations. To make this map we use the code Synrad3D developed
at  Cornell which generates and
tracks synchrotron-radiation photons in an accelerator beam line, including
specular and diffuse reflection on the chamber surface.  The photons are
generated randomly in any bending field, with initial
parameters determined by the local beam distribution, the local electromagnetic
field, and by the beam energy.  When a photon hits the chamber wall its
reflection probability depends on the energy and angle of incidence, as well as
on the material, including combinations of multiple layers, and on the surface
roughness. 

The first chapter of this thesis is devoted to the most basic introduction to
accelerator design \rojo\{a lo mejor accelerator design no es la mejor forma de
decirlo, basics of accelerators o algo asi, semblance quizá\} and then it presents some features
\rojo{a lo mejor en vez de some features, usar the description} of the
highest-energy circular colliders where our analysis was implemented.

We then present in the second chapter a qualitative description of S.R. \rojo\{a
lo mejor poner explícito synchrotron radiation\} for a further understanding of
it, the author recommends going through chapters 1, 2 and 3 of "Synchrotron
Radiation" by A. A. Sokolov. \rojo\{aqui no se si debo decir eso o no\ldots{} en caso
de que sí, ver como hacer la referencia de forma correcta\}

A description of the code Synrad3D is given in the third chapter. This is the
code we use for all simulations described in this thesis. Also, some tests and
benchmarking of the code will be mentioned.

Finally, in the last two chapters, we present our simulations, and the
implications they have to the previously mentioned accelerators.

\chapter{Accelerators}
\label{sec:orgb7f1f1f}
\section{LHC}
\label{sec:org049c567}
\begin{enumerate}
\item HL-LHC
\label{sec:org3c7bf75}
\item HE-LHC
\label{sec:orgeed56d8}
\end{enumerate}
\section{FCC-hh}
\label{sec:org23b9199}
\chapter{Synchrotron Radiation}
\label{sec:org5e1a041}
\chapter{Synrad3D}
\label{sec:org121f83e}
\chapter{Results}
\label{sec:org7dd8dd8}
\chapter{Conclusions}
\label{sec:org8164c45}
aqui van als conclusiones
\chapter{References}
\label{sec:orgaf0569c}
\end{document}
