% Chapter Template

\chapter*{Conclusion} % Main chapter title
\addcontentsline{toc}{chapter}{Conclusion}
\label{conclusion} % Change X to a consecutive number; for referencing this chapter elsewhere, use \ref{ChapterX}

\lhead{\emph{Conclusion}} % Change X to a consecutive number; this is for the header on each page - perhaps a shortened title
%----------------------------------------------------------------------------------------
%	SECTION 1
%----------------------------------------------------------------------------------------

No big difference between the Sawtooth Pattern beam screen and the smooth one were observed in both the position in the transversal direction (s) and energy of the absorbed photons (Figures \ref{fig:s}$-$\ref{fig:energia}). The difference that do appear, are only statistical error. I strongly believe this is because none of the photons bounced off the screen.

The simulation considering a sawtooth pattern in the beam screen showed a more extended distribution on the X-Y map Figure \ref{fig:xy} than the one on smooth screen \ref{fig:xyplana}. This can make the cooling process more efficient.

Considering all this, the main conclusion is that the length we need to consider in our simulations should be longer than threee elements to really appreciate the difference the sawtooth pattern on the beam screen makes. But this simulation does help us to notice that in small accelerators having a sawtooth pattern on te beam screen would help us a little to nothing.
\section*{Future Work} % Main chapter title
\addcontentsline{toc}{section}{Future Work}
\label{FW}
The first step is to share our result with people at CERN, and compare with measurements. 
There are also two further steps that must be taken in the future to complete this work.
The first one is to take an arc section long enough for photons to bounce at least two times in order to make the differences more noticeable. 
The second one is to take the whole arc into account, this will help us pinpoint the sections that suffer more heating due to synchrotron radiation and will help us fully understand how the sawtooth pattern help us distributing the heat produced by SR and to reduce the primary emmision of electrons that would give rise to an electron cloud. And then back to sharing the resuts with CERN and comparing to measurements.