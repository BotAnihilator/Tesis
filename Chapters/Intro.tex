% Chapter 1

\chapter*{Introduction} % Main chapter title
\addcontentsline{toc}{chapter}{Introduction}
\label{intro} % For referencing the chapter elsewhere, use \ref{Chapter1} 

\lhead{Chapter 1. \emph{Introduction}} % This is for the header on each page - perhaps a shortened title


Synchrotron Radiation (SR) emission is the phenomenon
product of the change of direction of relativistic-charged particles.  Ever
since it was observed in 1947 at the GE Synchrotron, \azul{cite required} the
interest in its study has increased, even though great part of the theory was
developed before that, and it is still being developed at the present (after a
hundred years).

In electron storage rings and circular accelerators, the emitted synchrotron
radiation reaches high intensities and energies. Although, Nowadays, there are
dedicated electron accelerators with the purpose of producing SR with specific
characteristics; in many other accelerators, SR is considered a secondary
product which is nocuous to the operation of the machine.

The power of emitted SR depends on the square of the Lorentz factor
(\(\gamma_r^{2}\)). For this reason, SR was considered negligible in hadron
accelerators. It was until the construction of the Large Hadron Collider (LHC),
that hadron SR became a problem for protons with a \(\gamma_r^{2}\geq 7460\)
that the negative effects of SR were not negligible. The LHC is the coldest
place in the universe. In order to keep the superconductor magnetic coils at 1.9
°K, gigantic cryogenic plants are used. One of the main heat loads of the arcs
is SR with a power deposition of 0.17 W/m/aperture. The vacuum in the LHC is as
low as interestellar vacuum $\sim$\SI{e-14}{bar}. High-energy photons from SR can
remove molecules and particles from the wall degrading this vacuum.  Yet another
problem related to SR is the formation of electron clouds (EC). Such clouds
could grow from seed electrons torn from the wall by incident photons. These
electrons are attracted and accelerated by the field of the proton beam and
also, follow the magnetic field lines inside the accelerator magnets. When
accelerated electrons hit the vacuum chamber they can generate second electrons
leading to an avalanche build up of electrons. Electron-cloud problems are the
main motivation for the studies of this thesis.

The main objective of this work is to map the absorption points of SR photons
in the arcs of the LHC, with the idea that this map could be used in the future
to generate a photon distribution function (PDF) to be used as an input for
electron-cloud simulations. To make this map we use the code Synrad3D developed
at Cornell which generates and tracks synchrotron-radiation photons in an
accelerator beam line, including specular and diffuse reflection on the chamber
surface.  The photons are generated randomly in any bending field, with initial
parameters determined by the local beam distribution, the local electromagnetic
field, and by the beam energy.  When a photon hits the chamber wall its
reflection probability depends on the energy and angle of incidence, as well as
on the material, including combinations of multiple layers, and on the surface
roughness.

The first chapter of this thesis is devoted to the most basic introduction to
\azul{accelerators and its design always prioritizing circular accelerators.}
Then it presents some features of the highest-energy circular colliders for
which our analysis was implemented.

In the second chapter we present a qualitative description of SR and its
relation to EC build-up. 

A description of the code '\srthree{}' is given in the third chapter. \srthree
is the code we use for all simulations described in this thesis. Also, some
tests and benchmarking of the code will be mentioned.

Finally, in the last two chapters, we present our simulation results, and the
implications they have for CERN's present and future highest-energy circular accelerators. 


%----------------------------------------------------------------------------------------

%\section{sexion}
