%%% Local Variables:
%%% mode: latex
%%% TeX-master: "Tesis"
%%% End:

%----------------------------------------------------------------------------------------
%	ABSTRACT PAGE
%----------------------------------------------------------------------------------------

\addtotoc{Abstract} % Add the "Abstract" page entry to the Contents

\abstract{\addtocontents{toc}{\vspace{1em}} % Add a gap in the Contents, for aesthetics

  At energies in the order of TeV, synchrotron radiation (SR) is very high, even
  in a hadron beam. SR could be regarded as an important heat load to the
  cryogenic system cooling the superconducting electromagnets. In this work SR
  is simulated using \srthree to analize which lattice elements of a section of
  the LHC absorb the most photon and analize how does it affect to have a
  sawtooth pattern in the beam screen.  }

\addtotoc{Resumen} % Add the "Abstract" page entry to the Contents
\selectlanguage{spanish}
\abstract{\addtocontents{toc}{\vspace{1em}} % Add a gap in the Contents, for aesthetics
  A energías en el orden de TeV la radición de sincrotrón es muy alta, incluso
  usando rayos de hadrones, esta radiación puede representar una carga calórica
  demasiado grande para electroimanes trabajando en estado criogénico. En este
  trabajo se simula la radiación de sincrotrón generada por un haz de protones
  en el LHC, utilizando \srthree, para analizar qué sección en un arco absorben
  la mayor cantidad de fotones y analizar como nos afecta tener en la pantalla
  ('beam screen') un patrón de diente de sierra.  } \selectlanguage{english}
\clearpage % Start a new page
